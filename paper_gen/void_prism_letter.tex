\documentclass[fleqn,usenatbib]{mnras}
\usepackage[T1]{fontenc}
\usepackage{graphicx}
\usepackage{amsmath}
\usepackage{xspace}
\usepackage{booktabs}

\newcommand{\dLPD}{\ensuremath{\Delta\mathrm{LPD}}\xspace}
\newcommand{\EG}{\ensuremath{E_{\mathrm{G}}^{\mathrm{void}}}\xspace}

\title[Void-Prism Signal Status]{Void-Prism Ancillary Signal Status: Persistent Directionality Without Decisive Null Rejection}

\author[Simulation Station Collaboration]{
Simulation Station Collaboration\\
E-mail: contact@simulationstation.example
}

\date{Accepted 2026 February 12. Received 2026 February 12; in original form 2026 February 12}
\pubyear{2026}

\begin{document}
\maketitle

\begin{abstract}
We present a standalone release of the void-prism ancillary pipeline and a compact evidence update using five independent posterior seeds.
The joint scorer compares modified-gravity embeddings against an internal GR baseline via \dLPD.
Across the explicit multi-embedding run (minimal, slip-allowed, screening-allowed), all seeds remain same-sign positive, with mean fitted-amplitude shifts of
$+0.0423$ (minimal), $+0.0414$ (slip-allowed), and $+0.0490$ (screening-allowed).
To test whether this persistence is likely to reflect a real isolated signal, we run an efficient battery: block-permutation nulls, block-sign nulls, leave-one-block-out (LOO), and split-consistency checks.
The sign persistence remains strong, but null-tail probabilities are not small (typical upper-tail $p\sim0.27$--$0.56$), and LOO minima are negative in all embeddings.
We therefore interpret the current void-prism channel as directionally consistent but not yet decisive: the signal persists, but statistical fluke or residual systematics remain plausible.
\end{abstract}

\begin{keywords}
cosmology: observations -- cosmology: theory -- large-scale structure of Universe -- gravitational lensing: weak
\end{keywords}

\section{Introduction}
Ancillary probes are useful for stress-testing the direction and channel localization of primary cosmological anomalies.
Here we summarize the status of a void-conditioned prism observable based on the ratio
\begin{equation}
\EG(\ell) \propto \frac{C_{\kappa,v}(\ell)}{C_{\theta,v}(\ell)},
\end{equation}
constructed from Planck lensing, ACT/SDSS-derived velocity-proxy maps, and BOSS DR12 void catalog splits.
The score reported throughout is
\begin{equation}
\dLPD_{\rm vs\,GR} \equiv \mathrm{LPD}_{\rm model} - \mathrm{LPD}_{\rm GR},
\end{equation}
with GR treated as an internal baseline generated from the same background draws.

\section{Data Products and Pipeline}
The release includes a standalone implementation containing:
(i) theta-map construction,
(ii) z/Rv-binned suite measurement with jackknife covariance, and
(iii) joint posterior-predictive scoring.
The bundled suite uses 8 blocks (4 redshift bins each split by median void radius), with 48-dimensional concatenated data vector and jackknife covariance from 603 regions.

Two result families are included:
\begin{enumerate}
\item A legacy 5-seed minimal-embedding run with mean $\dLPD_{\rm vs\,GR}=+0.0182$.
\item An explicit 5-seed $\times$ 3-embedding run with non-degenerate settings
($\eta_0=1.12$, $\eta_1=-0.18$, $\alpha_{\rm env}=0.25$, $\mu_{\rm P,high\text{-}z}=1.05$).
\end{enumerate}

\section{Updated Multi-Embedding Results}
Figure~\ref{fig:deltas} shows per-seed \dLPD values for each embedding in the explicit run.
All seeds are positive in all embeddings, with modest seed-to-seed scatter.
Screening-allowed gives the largest mean shift, while minimal and slip-allowed are close.

\begin{figure}
\centering
\includegraphics[width=\columnwidth]{fig_delta_by_seed.pdf}
\caption{Per-seed fitted-amplitude \dLPD values in the explicit 5-seed multi-embedding run. Dashed lines show embedding means.}
\label{fig:deltas}
\end{figure}

\section{Fast Signal Battery}
To test whether persistence alone indicates a real isolated signal, we ran a low-cost battery over all five seeds and three embeddings (draw cap 256):
\begin{itemize}
\item block-permutation null (misalignment null),
\item block-sign null,
\item leave-one-block-out robustness,
\item split consistency (low/high-$z$, small/large-$R_v$).
\end{itemize}

Figure~\ref{fig:pvals} shows null-tail behavior.
Upper-tail permutation $p$-values cluster near $0.27$--$0.30$ and sign-null $p$-values near $0.53$--$0.56$, i.e. not extreme.
Figure~\ref{fig:robust} shows that split means remain positive, but LOO minima are negative for all embeddings.

\begin{figure}
\centering
\includegraphics[width=\columnwidth]{fig_battery_pvalues.pdf}
\caption{Fast battery null diagnostics across five seeds for each embedding. Horizontal red lines mark reference levels 0.05 and 0.01.}
\label{fig:pvals}
\end{figure}

\begin{figure}
\centering
\includegraphics[width=\columnwidth]{fig_robustness_checks.pdf}
\caption{Robustness diagnostics. Left: mean and scatter of LOO minimum \dLPD (negative values indicate block fragility). Right: mean split \dLPD values remain positive.}
\label{fig:robust}
\end{figure}

\section{Interpretation and Conclusion}
The updated status is:
\begin{enumerate}
\item \textbf{Persistence:} the direction is stable (all seeds positive in all tested embeddings).
\item \textbf{Non-decisiveness:} efficient nulls are not strongly rejected.
\item \textbf{Fragility:} negative LOO minima indicate sensitivity to block composition.
\end{enumerate}

Therefore, the void-prism channel remains a useful directional cross-check but not a standalone discriminator at current S/N and calibration depth.
The practical reading is concise: the signal persists, but a statistical fluke or residual systematic explanation is still plausible.

\section*{Data Availability}
All artifacts used in this letter are included in the repository under
\texttt{artifacts/ancillary/void/}, including the explicit multi-embedding run and battery outputs.

\section*{Acknowledgements}
This manuscript was generated from the public repository release of the void-prism ancillary pipeline.

\begin{thebibliography}{99}

\bibitem[Planck Collaboration et al.(2020)]{Planck2018Lensing}
Planck Collaboration et al., 2020, A\&A, 641, A8

\bibitem[Aiola et al.(2020)]{Aiola2020}
Aiola S. et al., 2020, JCAP, 12, 047

\bibitem[Mao et al.(2017)]{Mao2017}
Mao Q. et al., 2017, MNRAS, 466, 1042

\end{thebibliography}

\bsp
\label{lastpage}
\end{document}
