\documentclass[fleqn,usenatbib]{mnras}
\usepackage[T1]{fontenc}
\usepackage{graphicx}
\usepackage{amsmath}
\usepackage{amssymb}
\usepackage{xspace}
\usepackage{booktabs}

\newcommand{\dLPD}{\ensuremath{\Delta\mathrm{LPD}}\xspace}
\newcommand{\EG}{\ensuremath{E_{\mathrm{G}}^{\mathrm{void}}}\xspace}

\title[Void-Prism Signal Update]{Void-Prism Ancillary Signal Update: Persistent Direction with Sub-1$\sigma$ Null-Calibrated Evidence}

\author[Simulation Station Collaboration]{
Simulation Station Collaboration\\
E-mail: contact@simulationstation.example
}

\date{Accepted 2026 February 12. Received 2026 February 12; in original form 2026 February 12}
\pubyear{2026}

\begin{document}
\maketitle

\begin{abstract}
We present a compact status update of the standalone void-prism ancillary pipeline using five independent posterior seeds and three modified-gravity embeddings.
The joint score is \dLPD relative to an internal GR baseline.
All seeds remain positive in all embeddings, with mean fitted-amplitude shifts
$+0.0423$ (minimal), $+0.0414$ (slip-allowed), and $+0.0490$ (screening-allowed).
A sign-only persistence test gives one-sided $p=0.03125$ ($1.86\sigma$), but null-calibrated tests are not extreme:
fast battery permutation/sign/data-placebo means are $p\sim0.27$--$0.55$, map-level rotation/random-center placebos are $p\sim0.47$--$0.57$, corresponding to roughly $|Z|\lesssim0.6$ one-sided.
Robustness remains limited: leave-one-out and leave-two-out all-positive fractions are zero for all embeddings, with nonpositive subset fractions $0.25$ and $0.321$.
An external sky split (NGC versus SGC) is same-sign positive in all embeddings but amplitude-mismatched by factors $\sim15$--$16$.
The current channel is therefore directionally consistent but not decision-grade.
\end{abstract}

\begin{keywords}
cosmology: observations -- cosmology: theory -- large-scale structure of Universe -- gravitational lensing: weak
\end{keywords}

\section{Introduction}
Ancillary probes are useful for stress-testing the direction and channel localization of primary cosmological anomalies.
Here we summarize the status of a void-conditioned prism observable based on the ratio
\begin{equation}
\EG(\ell) \propto \frac{C_{\kappa,v}(\ell)}{C_{\theta,v}(\ell)},
\end{equation}
constructed from Planck lensing, ACT/SDSS-derived velocity-proxy maps, and BOSS DR12 void catalog splits.
The score reported throughout is
\begin{equation}
\dLPD_{\rm vs\,GR} \equiv \mathrm{LPD}_{\rm model} - \mathrm{LPD}_{\rm GR},
\end{equation}
with GR treated as an internal baseline generated from the same background draws.

\section{Data Products and Pipeline}
The release includes a standalone implementation with:
(i) theta-map construction,
(ii) z/Rv-binned suite measurement with jackknife covariance, and
(iii) joint posterior-predictive scoring.
The bundled suite uses 8 blocks (4 redshift bins each split by median void radius), with 48-dimensional concatenated data vector and jackknife covariance from 603 regions.
To support long runs, the current release includes partial-checkpoint/resume support for jackknife measurement and joint scoring.

The evidence update summarized here uses:
\begin{enumerate}
\item explicit 5-seed $\times$ 3-embedding joint scoring with non-degenerate settings
($\eta_0=1.12$, $\eta_1=-0.18$, $\alpha_{\rm env}=0.25$, $\mu_{\rm P,high\text{-}z}=1.05$);
\item a fast battery with 500 permutation/sign/data-placebo draws;
\item map-level placebo tests (rotation and mask-randomized centers, 64 draws per mode);
\item NGC/SGC split replication at max-draws 5000.
\end{enumerate}

\section{Updated Multi-Embedding Results}
Figure~\ref{fig:deltas} shows per-seed \dLPD values for each embedding in the explicit run.
All seeds are positive in all embeddings, with modest seed-to-seed scatter.
Screening-allowed gives the largest mean shift, while minimal and slip-allowed are close.

\begin{figure}
\centering
\includegraphics[width=\columnwidth]{fig_delta_by_seed.pdf}
\caption{Per-seed fitted-amplitude \dLPD values in the explicit 5-seed multi-embedding run. Dashed lines show embedding means.}
\label{fig:deltas}
\end{figure}

\section{Null-Calibrated Tests}
To test whether persistence indicates an isolated signal, we ran a fast battery over all five seeds and three embeddings:
\begin{itemize}
\item block-permutation null (misalignment null),
\item block-sign null,
\item data-placebo permutation null,
\item leave-one and leave-two block drop tests,
\item split consistency (low/high-$z$, small/large-$R_v$).
\end{itemize}

Figure~\ref{fig:pvals} and Table~\ref{tab:nulls} summarize the null tails.
Permutation and data-placebo tails are typically $p\sim0.27$--$0.31$; sign-null tails are $p\sim0.51$--$0.55$.
Under one-sided Gaussian mapping, this corresponds to approximately $|Z|\lesssim0.6$.
Map-level placebos (Figure~\ref{fig:mapplacebo}) are also non-extreme.

\begin{figure}
\centering
\includegraphics[width=\columnwidth]{fig_battery_pvalues.pdf}
\caption{Fast-battery null diagnostics across five seeds for each embedding: permutation, block-sign, and data-placebo permutation tests. Horizontal red lines mark reference levels 0.05 and 0.01.}
\label{fig:pvals}
\end{figure}

\begin{table}
\centering
\caption{Mean upper-tail null $p$-values by embedding (five seeds).}
\label{tab:nulls}
\begin{tabular}{lccccc}
\toprule
Embedding & Perm & Sign & Data-placebo & Rotate & Random-mask \\
\midrule
minimal & 0.289 & 0.521 & 0.307 & 0.471 & 0.551 \\
slip\_allowed & 0.271 & 0.511 & 0.302 & 0.486 & 0.569 \\
screening\_allowed & 0.303 & 0.549 & 0.293 & 0.471 & 0.511 \\
\bottomrule
\end{tabular}
\end{table}

\begin{figure}
\centering
\includegraphics[width=\columnwidth]{fig_map_placebo_pvalues.pdf}
\caption{Map-level placebo diagnostics (global rotations and mask-randomized centers). Mean tails remain non-extreme for all embeddings.}
\label{fig:mapplacebo}
\end{figure}

\section{Robustness and Split Replication}
Figure~\ref{fig:robust} shows two robustness diagnostics.
All embeddings keep positive low/high-$z$ and small/large-$R_v$ split means, but block-drop stability is weak:
the fraction of nonpositive subsets is $0.25$ for leave-one-out and $0.321$ for leave-two-out (all embeddings).
The worst leave-two-out pair is identical in all runs:
\texttt{zbin0\_small\_z0.200-0.360} plus \texttt{zbin1\_small\_z0.360-0.480}.

\begin{figure}
\centering
\includegraphics[width=\columnwidth]{fig_robustness_checks.pdf}
\caption{Robustness diagnostics. Left: fraction of nonpositive block-drop subsets (LOO and L2O). Right: intra-suite split means.}
\label{fig:robust}
\end{figure}

External split replication (Figure~\ref{fig:splitrepl}, Table~\ref{tab:split}) is same-sign positive but strongly amplitude-mismatched: SGC/NGC ratios are $\sim15$--$16$ in all embeddings.

\begin{figure}
\centering
\includegraphics[width=\columnwidth]{fig_split_replication.pdf}
\caption{NGC/SGC split replication. Left: per-split means. Right: amplitude ratio (SGC/NGC), with dashed line at factor 2.}
\label{fig:splitrepl}
\end{figure}

\begin{table}
\centering
\caption{Split replication mean \dLPD and SGC/NGC amplitude ratio.}
\label{tab:split}
\begin{tabular}{lccc}
\toprule
Embedding & NGC mean & SGC mean & Ratio \\
\midrule
minimal & +0.964 & +15.519 & 16.10 \\
slip\_allowed & +0.922 & +15.126 & 16.41 \\
screening\_allowed & +1.121 & +16.962 & 15.13 \\
\bottomrule
\end{tabular}
\end{table}

\section{Conclusion}
The current status is:
\begin{enumerate}
\item \textbf{Directional persistence:} all seeds are positive in all tested embeddings.
\item \textbf{Weak null-calibrated significance:} null tails correspond to about sub-1$\sigma$ evidence.
\item \textbf{Structural fragility:} block-drop tests and NGC/SGC amplitude mismatch indicate incomplete robustness.
\end{enumerate}

Therefore, the void-prism channel remains useful as a directional cross-check, but not yet as a standalone discriminator.
The data currently favor persistence in sign, while a fluke or residual systematic interpretation remains viable.

\section*{Data Availability}
All artifacts used here are in-repo under:
\texttt{artifacts/ancillary/void/void\_prism\_joint\_explicit\_multiembed\_20260212\_001503UTC/},
\texttt{outputs/void\_prism\_signal\_battery\_l2o\_placebo\_20260212\_rerun/},
\texttt{outputs/void\_prism\_map\_placebo\_20260212\_64x2/},
and \texttt{outputs/void\_prism\_split\_replication\_20260212\_5000/}.

\section*{Acknowledgements}
This manuscript was generated from the public repository release of the void-prism ancillary pipeline.

\begin{thebibliography}{99}

\bibitem[Planck Collaboration et al.(2020)]{Planck2018Lensing}
Planck Collaboration et al., 2020, A\&A, 641, A8

\bibitem[Aiola et al.(2020)]{Aiola2020}
Aiola S. et al., 2020, JCAP, 12, 047

\bibitem[Mao et al.(2017)]{Mao2017}
Mao Q. et al., 2017, MNRAS, 466, 1042

\end{thebibliography}

\bsp
\label{lastpage}
\end{document}
